\cventry{2021--}{Independent Researcher}{Rakta Network Oy}{Finland}{}{ 
\begin{itemize}
    \item Offer consulting and software development services for mathematical and scientific applications, specializing in imaging, image processing, and analysis for large-scale volumetric data.
\end{itemize}}

\cventry{2018--2021}{Research Scientist}{University of California}{San Francisco}{}{ 
\begin{itemize}
    \item Provided computational support for internal and collaborative research projects at the National Center for X-ray Tomography.
    \item Maintained and developed the automated image processing pipeline.
    \item Integrated modern machine learning techniques into soft X-ray tomography
\end{itemize}}
\cventry{2016--2018}{Postdoctoral Scholar}{University of California}{San Francisco}{}{ 
\begin{itemize}
    \item Managed the image processing pipeline at the National Center for X-ray Tomography
    \item Worked on transforming raw images into final volumetric representations for both fluorescence and X-ray microscopes, ensuring high-quality results for scientific analysis.
    \item  Developed and integrated a fully automatic alignment procedure into the image processing pipeline, reducing the time from acquisition to visualization from approximately 30 hours to 5 minutes.
\end{itemize}}
%
\cventry{2011--2016}{Doctoral Student}{University of Jyväskylä}{Finland}{}{ 
\begin{itemize}
    \item Conducted extensive theoretical and numerical research on random deposition networks, focusing on the impact of steric hindrance (physical obstruction) between constituents.
    \item  Demonstrated that steric hindrance significantly influences the contact formation and statistical properties of these networks, even in dilute systems.
    \item  Advanced the understanding of how physical obstructions impact the connectivity and formation of contacts in random fiber networks.
\end{itemize}}
%
\cventry{2010--2011}{Research associate}{University of Jyväskylä}{Finland}{}{ 
\begin{itemize}
    \item  Utilized micro- and nano-scale X-ray CT to analyze paper and cardboard structures.
    \item  Developed and implemented quantitative analysis tools for 3D tomographic images.
    \item  Enhanced material characterization techniques through multidisciplinary collaboration.
\end{itemize}
}